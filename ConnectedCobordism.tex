\documentclass[a4paper]{article}
\usepackage[utf8]{inputenc}
\usepackage{hyperref}

\usepackage{enigtex-macros}
\usepackage{enigtex-cat}
\newcommand{\eniglanguageindex}{0} %TODO
\usepackage{enigthm}


\title{Connected cobordisms and their representations}
\author{Manuel \& ?}

\begin{document}

\maketitle

\section{Rough idea}

Connected sum is a commutative monoid on (isomorphism classes of) connected manifolds.
I'd even bet it's a symmetric monoidal structure on the category of connected manifolds.
But what about cobordisms as morphisms between manifolds?
How do we get a cobordism category where connected sum plays a role?
There are four possibilities:
\begin{itemize}
	\item Forget about connected sum, disjoint union is good enough.
		This gives standard TQFTs.
	\item Have connected sum only on the bordisms,
		and disjoint union on the boundaries.
		Studied in Section \ref{sec:connect bordisms}.
	\item Have connected sum mainly on the boundaries,
		but not in the interior of the bordisms
		(only in the collars).
		Studied in Section \ref{sec:connect boundaries}.
	\item Have connected sum on the boundaries and the bordisms.
		This requires a choice of interval on every bordism,
		along which we'll glue.
		Studied in Section \ref{sec:connect everything}.
\end{itemize}
So much choice!
And all of them are not inequivalent (at least monoidally).
But we'll see how some ideas are special cases of others,
in a certain sense.
And some ideas will stop working pretty quickly.

\section{Prerequisites}

\subsection{Connected sum}

\begin{definition}
	\label{def:connected sum}
	Take three $n$-manifolds $M_1, M_2, N$ with $\partial N \neq \emptyset$
	and (collared?) embeddings $i_1\colon N \hookrightarrow M_1$, $i_2\colon \overline{N} \hookrightarrow M_2$.
	The \textbf{connected sum} of $M_1$ and $M_2$ along $N$ (with the $i_{1,2}$ implicitly assumed),
	denoted as $M_1 \connsum_N M_2$,
	is the manifold $M_1 \backslash_{i_1} N \cup_{\partial N} M_2 \backslash_{i_2} N$.
	
	If $N = D^n$ (the $n$-disk) and the embeddings are standard,
	they are omitted and we're talking simply about the connected sum $M_1 \connsum M_2$.
\end{definition}

\begin{definition}
	An $n$-dimensional bordism $\Sigma\colon M_1 \to M_2$ is a quadruple $(\Sigma, M_1, M_2, \phi\colon \partial \Sigma \xrightarrow{\cong} M_1 \sqcup \overline{M_2})$,
	where $\Sigma$ is an $n$-manifold, $M_{1,2}$ are closed $(n-1)$-manifolds and $\phi$ is a collared diffeomorphism.
	
	For $\Sigma_1, \Sigma_2\colon M_1 \to M_2$,
	a diffeomorphism between the two bordisms is a diffeomorphism of the underlying manifolds $\Sigma_{1,2}$ that agrees with $\phi_{1,2}$.
	From now on, we will typically understand all bordism up to these diffeomorphisms.
\end{definition}
It is well known that bordisms form a symmetric monoidal category $\operatorname{Bord}_n$,
where the monoidal product is the disjoint union.

\section{Connect the bordisms}
\label{sec:connect bordisms}

\begin{definition}
	Let $\operatorname{CBord}_n$ be the following ``braided'' monoidal category-without-identities:
	\begin{description}
		\item[Objects]
			Closed (not necessarily connected) $(n-1)$-manifolds.
		\item[Morphisms]
			$\operatorname{CBord}_n(M_1, M_2)$ is the set of (diffeomorphism equivalence classes of) \emph{connected} bordisms.
			\begin{description}
				\item[Identities]
					Note that unconnected $(n-1)$-manifolds don't have identity morphisms.
					Connected ones have the cylinder as identity, as usual.
					It is of course possibly to call the connected sum of several cylinders an identity bordism,
					but it doesn't satisfy the equations an identity should satisfy.
				\item[Composition]
					Composition is gluing along the collared boundary.
			\end{description}
		\item[Monoidal product]
			On objects/boundaries, disjoint union.
			On morphisms/bordisms, connected sum (along $D^n$ in arbitrary points in the interior).
		\item[Braiding]
			The braiding is the connected sum of the identity bordisms,
			with a swap diffeomorphism on one end.
			It satisfies some sort of braid equation.
			The braiding is not symmetric, at least not in the sense in which ``identity bordisms'' were defined.
	\end{description}
\end{definition}
All in all, this thing is pretty broken.
And one wouldn't expect it to have any interesting new behaviour,
since we can model all of it in ordinary bordisms!
Simply replace every morphism in $\operatorname{CBord}_n$ with the same bordism,
but with a $D^n$ cut out.
When taking the disjoint union of two such bordisms,
just stick a maccaroni (the duality morphism $\coev\colon S^{n-1} \sqcup S^{n-1} \to \emptyset$) on top and you'll get the connected sum.

\medskip

% TODO
TODO: A discussion relating this idea to the fact that if $\dim \mathcal{Z}(S^{n-1}) = 1$,
then $\mathcal{Z}(\Sigma_1 \connsum \Sigma_2) = \frac{\mathcal{Z}(\Sigma_1) \cdot \mathcal{Z}(\Sigma_2)}{\mathcal{Z}(S^n)}$ for closed bordisms $\Sigma_{1,2}$.

\section{Connect the boundaries}
\label{sec:connect boundaries}

In Definition \ref{def:connected sum},
we can probably allow embedded manifolds with corners.
This gives the following definition:
\begin{definition}
	The manifold with corner $\{(x_0, x_1, \dots, x_n) \in \R^n | x_0 \geq 0, \sum x_i^2 \leq 1 \}$
	is called \textbf{half of $D^n$}.
	Its ``bottom`` is the $D^{n-1}$ given by $x_0 = 0$.

	Given two bordisms $\Sigma_1\colon M_{1,1} \to M_{1,2}$ and $\Sigma_2\colon M_{2,1} \to M_{2,2}$
	such that the $M_{i,j}$ are all connected and closed,
	chose two embeddings of half of $D^n$ in each of the bordisms such that the bottoms lie in the boundaries.
	Cut the half of $D^n$ out,
	and perform connected sum.
	This will give the usual connected sum on the boundaries,
	and a new, connected, bordism.
	
	This operation is called \textbf{connected sum on the boundaries}.
\end{definition}
\begin{definition}
	The not-quite-monoidal category $\operatorname{cBord}_n$ is defined as follows:
	\begin{description}
		\item[Objects]
			\emph{Connected}, closed $(n-1)$-manifolds.
		\item[Morphisms]
			\emph{Connected} cobordisms.
			Identity and composition are as usual.
		\item[``Monoidal product'']
			On objects, connected sum.
			The monoidal unit is now $S^{n-1}$.
			On morphisms, connected sum on the boundaries.
			
			Unfortunately, this doesn't send the monoidal product of two identities to an identity.
			It also doesn't respect the interchange law:
			If we first connect sum and then compose,
			there is a gluing point right in the middle of the bordism.
			If we compose first and then connect sum,
			that gluing point isn't there.
		\item[Braiding]
			The braiding consists of the identity bordism and the diffeomorphism $M_1 \connsum M_2 \cong M_2 \connsum M_1$.
			Looks pretty symmetric to me.
	\end{description}
\end{definition}
Again, this is quite a broken thing.

\section{Connect everything}
\label{sec:connect everything}

Things seem not to work if we put the connected sum just on bordisms or boundaries.
Maybe we have to do both simultaneously.
But when connect summing all the way along the whole bordism,
we have to sum along a whole interval,
and then the particular choice of interval is relevant.

\begin{definition}
	A \textbf{zippered bordism} $\Sigma\colon (M_0, p_0 \in M_0) \to (M_1, p_1 \in M_1)$ is a quintuple $(\Sigma, M_0, M_1, \phi, z\colon [0,1] \times D^{n-1} \to \Sigma)$,
	such that $(\Sigma, M_0, M_1, \phi)$ is a \emph{connected} bordism and $z$ (the ``zipper'') is a framed embedding of the interval such that $z(0) = \phi^{-1}(p_0)$ and $z(1) = \phi^{-1}(p_1)$.
	
	A diffeomorphism of zippered bordisms is required to preserve $z$.
	(Corollary: the endpoints must already agree.)
\end{definition}
Intuitively, the boundaries are pointed and there is a framed interval,
the zipper,
which runs from one point to the other, through the bordism.
\begin{definition}
	The \textbf{zippered connected sum} between two zippered bordisms is the connected sum along the embeddings given by the zippers.
	By globally choosing an arbitrary point on the boundary of $D^{n-1}$,
	we should be able to equip the connected sum with a zipper.
\end{definition}

\begin{definition}
	The symmetric monoidal category $\operatorname{ZBord}_n$ is defined as:
	\begin{description}
		\item[Objects]
			\emph{Connected, pointed}, oriented, closed manifolds.
		\item[Morphisms]
			\emph{Zippered}, oriented bordisms.
			When composing, note that the boundaries have to agree as pointed manifolds,
			and therefore the zippers can be composed.
			The identity is the cylinder.
		\item[Monoidal structure]
			On objects, connected sum.
			The monoidal unit is $S^{n-1}$.
			On morphisms, zippered connected sum.
% 			TODO
			To do: Check associators etc. carefully.
		\item[Braiding]
			The braiding is the cylinder with a diffeomorphism $(M_0, p_0) \connsum (M_1, p_1) \cong (M_1, p_1) \connsum (M_0, p_0)$ that leaves the point intact.
			Sounds symmetric to me.
	\end{description}
\end{definition}

\begin{remark}
	The essential idea is already present in \href{http://mathoverflow.net/questions/179344/what-are-tqfts-that-are-multiplicative-under-connected-sums-do-bordisms-with-co}{http://mathoverflow.net/questions/179344/what-are-tqfts-that-are-multiplicative-under-connected-sums-do-bordisms-with-co}.
	
	Also note the early work \href{https://arxiv.org/pdf/q-alg/9603017.pdf}{https://arxiv.org/pdf/q-alg/9603017.pdf} by Thomas Kerler.
\end{remark}

\section{Connected TQFTs}

\begin{definition}
	A \textbf{zippered}, or \textbf{connected} TQFT in $n$ dimensions with values in $\mathcal{C}$
	is a symmetric monoidal functor $\mathcal{Z}\colon \operatorname{ZBord}_n \to \mathcal{C}$.
\end{definition}

Let's try and find a presentation for some low dimensions.

\subsection{Dimension 1}

$\operatorname{ZBord}_1$ has only one object,
the point.
It is the monoidal unit,
so it won't generate any new objects.
It's also trivially pointed.

The only endomorphism is its identity,
which is trivially zippered.
Remember that manifolds with more than one component are not allowed.

So $\operatorname{ZBord}_1$ is the trivial symmetric monoidal category.

Its representations are very sparse.
Indeed, to a given symmetric monoidal category,
there is exactly one functor.

Dimension 1 is all about the point,
and it's quite pointless.
Compare it to TQFTs, where we already have dualisability in that dimension.

\subsection{Dimension 2}

There is still only one object, $S^1$.
It's also the monoidal unit.
(Remember, no disjoint union.)
Actually, for every point on $S^1$,
there is an object, but they're all isomorphic,
as we'll see shortly.

Now as morphisms, finally something happens.
An endomorphism of $S^1$ is a surface with 2 holes and an oriented curve (the zipper) from one hole to the other.
So there are already infinitely many of them!
There is no information in the zipper, since we can shift both endpoints around arbitrarily.
So morphisms are classified by their genus.

Note specifically that there is only one zipper on the cylinder:
All winding numbers are diffeomorphic to the trivial one,
possibly by a few Dehn twists (which are invisible if we don't extend).
So all $S^1$s are isomorphic,
no matter what the point is.

What are the relations from the monoidal structure?
When we glue two surfaces along their zipper,
we imagine a thickening around the zipper,
we paint the right hand boundary of the thickening one of the surfaces in, say, green,
we cut the thickenings out,
we glue the surfaces on the boundaries of the thickenings
and finally we call the green paint the new zipper.
The number of holes clearly added,
so morphisms are freely generated by the doubly punctured torus,
an endomorphism of $S^1$.

\begin{proposition}
	Zippered 2d TQFTs are classified by an (arbitrary) endomorphism of the monoidal unit of the target category.
	(And by Eckmann-Hilton, its composite with itself is the same as its product with itself.)
	Still quite tame.
	(tame = lame)
\end{proposition}

\subsection{Dimension 3}

Wowowowoh!
3-manifolds are serious beasts!
Don't expect a complete classification just yet!

Objects are pointed surfaces.
They're classified by their genus,
and the tensor product adds the number of holes.
The point doesn't matter up to isomorphism.

Morphisms are connected bordisms.
If I'm not mistaken,
you can just fill a surface up and then do surgery on the interior.
The surgery can be visualised as a Heegard diagram,
and the zipper is a framed interval in there (probably).
Now it seems like the choice of zipper can matter when gluing stuff together.

There should be a decent string net TQFT definable on there.
Good luck classifying these things.
There are some theorems though that say how 3-manifolds can be decomposed into direct sums.
Still, lens spaces will account for infinitely many generators.

\section{Why?}

Good question.
Connected sum is like composition of morphisms, so one shouldn't expect

\subsection{Extended connected TQFTs?}

Already the 3d example was not classifiable.
Usually, to get better presentations, one resorts to extended TQFTs.
How to do this here?
I'd like to know.

% TODO Possibly supplement with a few well-known facts on connected sum & TQFTs (CY, the S^n trick, direct sum decomposition) and put it on Arxiv (as a negative result?)

\end{document}
